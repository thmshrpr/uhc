\documentclass{beamer}
\usepackage{beamerthemesplit}

\usepackage{graphicx}
\usepackage{epstopdf}
\usepackage{subfigure}

\usepackage{multirow}
\usepackage{amsfonts}
\usepackage{amsmath}

\title[GRIN]{Grin Optimizing Transformations}
\author[]{Aris \& Gideon}
\institute[]{Efficient Implementation of Functional Languages Seminar 2006}
%abusing the institute field.

\begin{document}

\section{New Horizons}
\begin{frame}
The identity transformation
\end{frame}

\section{Boquists Transformations}
\begin{frame}
\frametitle{Already implemented transformations}
Many optimizing transformations were already implemented:
Copy propagation (\texttt{CopyPropagation.cag})
Trivial case elimination (\texttt{CaseElimination.cag})
Sparse case elimination (\texttt{SparseCase.cag})
Constant Propagation (\texttt{})

Dead code elimination

\end{frame}

\begin{frame}
\frametitle{Ueber Eval/Apply Inline}
Evaluated case elimination

Update elimination
WHNF update elminiation
\end{frame}

\begin{frame}

Possible transformations
Late inlining
Case hoisting
Common subexpression elimination

Impossible transformations
Generalized unboxing
Case copy propagation
Arity raising

\end{frame}

\section{Simon Peyton-Jones' Inliner}
\begin{frame}
Boquist only inlines functions on two criteria (see p. 137):
\begin{enumerate}
\item called only once (and non-recursively)
\item size is smaller than a certain limit
\end{enumerate}

etc...
\end{frame}

\end{document}
